\begin{textblock}{\mycolwidth}(\rightpos,2.5)
\LHead{The Theory} \\
\large
There is a directed graph $G = (V, E)$ and a set of edge labels:
\begin{multicols}{2}
\normalsize
In a \emph{functional routing network} the edge labels are
functions $f$ on a semilattice $(S, \oplus, \leq)$, which can be
\begin{itemize}[itemindent=1cm,labelsep=0.5cm]
  \item \emph{extensive}, if $\id_S \leq f$,
  \item \emph{right-inclining}, if $g \leq f \circ g$ for all $g \in F$,
  \item \emph{left-inclining}, if $f \leq f \circ g$ for all $g \in F$,
  \item \emph{distributive}, if $f \circ (g \oplus h) = (f \circ g) \oplus (f \circ h)$
  for all $g, h \in F$.
\end{itemize}
In an \emph{algebraic routing network} the edge labels are elements from a semiring structure $(S, \oplus, \otimes, \leq, \nul, \eins)$, for example a \emph{bounded dioid}:
\begin{itemize}[itemindent=1cm,labelsep=0.5cm]
  \item $(S, \oplus, \otimes, \nul, \eins)$ is a semiring
  \item $(S, \oplus, \leq)$ is a semilattice,
  \item $\oplus$ is idempotent, i.e.\ $a \oplus a = a$, and
  \item $\eins \oplus a = a \oplus \eins = \eins$.
\end{itemize}
\end{multicols}
Problem: Given a starting value, find best paths between two vertices with respect to the partial order $\leq$ of the semilattice.

\emph{Profile search}: Find best paths for all starting values.

\textbf{Theorem about profile searches}:
The profile searches on functional routing networks with
\emph{extensive, inclining and distributive}
functions are algebraic routing problems on \emph{bounded dioids}.
\end{textblock}

\begin{textblock}{\mycolwidth}(\rightpos,8.2)
\LHead{Algebraic Abstractions} \\
\large
An idea from static analysis of programs \cite{cousot1977}:
\begin{itemize}[itemindent=2cm,labelsep=1cm]
  \item Abstraction $\alpha: S \to S'$
  \item Concretization $\gamma: S' \to S$
\end{itemize}
\begin{center}
\begin{tikzpicture}[scale=4]
\node[] (x) at (0, 0) {$0.43$};
\node[] (a2) at (-3.5, 0) {$4$};
\node[] (a1) at (3.5, 0) {$1$};
\node[] (fx) at (0, -1.5) {$0.7 \leq 0.74 \leq 2$};
\node[] (fa2) at (-3.5, -1.5) {$7$};
\node[] (fa1) at (3.5, -1.5) {$2$};
\draw[->] (x) -- (a2) node[midway, above]
  {$\alpha^\downarrow(a) = \lfloor 10 a \rfloor$};
\draw[->] (x) -- (a1) node[midway, above]
  {$\alpha^\uparrow(a) = \lceil a \rceil$};
\draw[->] (x) -- (fx) node[midway, right] {$+ 0.31$};
\draw[->] (a1) -- (fa1) node[midway, left] {$+1$};
\draw[->] (a2) -- (fa2) node[midway, right] {$+3$};
\draw[->] (fa1) -- (fx) node[midway, below]
  {$\gamma^\uparrow(a) = a$};
\draw[->] (fa2) -- (fx) node[midway, below]
  {$\gamma^\downarrow(a) = \frac{a}{10}$};
\end{tikzpicture}
\end{center}
Requires edge weighting functions to be isotonic: $s \leq t \to f(s) \leq f(t)$ for all $s, t \in S$ (also called \emph{FIFO} or \emph{non-overtaking}-property).

\textbf{Theorem about interval abstractions}: Routing abstractions yield valid lower and upper bounds on functional routing problems with isotonic functions.
\end{textblock}

\begin{textblock}{\mycolwidth}(\rightpos,15.5)
\LHead{Conclusions and Future Work} \\
\large
First tests yielded reasonable results, but we just started to implement
various shortest path algorithms for the state-based routing problem and
its profile version. Hence runtime and space analyzations are near future
goals. The model itself is promising, comprising time-dependent routing,
battery constraints, stochasticity and multi-criteria routing, but the
running time is an important question to answer soon.

In the long run we aim to extend GreenNav by more sophisticated algorithms
and services. An egoistic routing algorithm is but the first step towards
ecological sustainability. Multi-modality and stochasticity are two important
points to consider.
\end{textblock}